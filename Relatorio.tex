\documentclass{article}
\usepackage[utf8]{inputenc}

\title{Relatório - Fase II}
\author{
    Germano Huning Neuenfeld 9298340
    \\
    Lucas Moreira Santos 9345064
    \\
    Victor Wichmann Raposo 9298020
}
\date{Maio 2016}

\begin{document}

\maketitle
\section{Introdução}
O documento visa descrever o projeto do jogo Spacewar.
Allegro ideia e como baixar e como rodar 
Modelo ideia
\section{Classes}
Fizemos o uso de orientação a objetos usando structs. Definimos algumas classes que serão explicadas abaixo
cuja documentação de cada função está especificada no cabeçalho das respectiva classe.

\subsection{Body}
O Body funciona como base para outros corpos. Ele possuí propriedades básicas que todo corpo (nave, planeta, projétil) possui.
Os atributos implementados nele são: raio (double), peso (double), ângulo em radianos (double), posição (Vector), força (Vector) e velocidade (Vector).

\subsection{Projectile}
Essa é a representação dos projéteis. Os atributos que eles possuem são: corpo (Body) e duração (double).
Como não há orientação a objetos em C e portanto não há hierarquia, inserimos o corpo como atributo da struct.

\subsection{Ship}
Essa é a representação das naves. Os atributos que eles possuem são: corpo (Body) e nome (array de chars).
Como não há orientação a objetos em C e portanto não há hierarquia, inserimos o corpo como atributo da struct.

\subsection{Vector}
Essa é a representação de um vetor no $V^2$. Os atributos que eles possuem são: x (double) e y (double).

\section{MVC}
Utilizamos a arquitetura de software Model-View-Controller para desenvolvimento do projeto por facilitar o desenvolvimento em módulos.
\subsection{Modelos}
Procuramos implementar uma orientação a objetos no nosso projeto pois acreditamos que fosse simplificar o desenvolvimento e manutenção do
código. \\
 O objeto Vector que criamos simplificou muito o manuseio e cálculo de velocidades, acelerações e posições. \\
O arquivo "simulation.c" é responsável por aplicar a modelagem física, computando as forças e movimentos.
\subsubsection{Gravidade}
A cada interação zeramos os vetores de força de todos os corpos e depois aplicamos as forças gravitacionais entre:
\begin{itemize}
\item Planeta e nave 1;
\item  Planeta e nave 2;
\item Nave 1 e nave 2;
\end{itemize}
Propositalmente, não computamos a gravidade nos projeteis pois queremos que seu movimento seja retilínio.

\subsubsection{Movimento}
Assumimos que, para o dt dado, o movimento dos corpos é um movimento uniformemente variado. A partir disso calculamos o espaço e a velocidade dos corpos em relação ao eixo x e ao eixo y;

\subsubsection{Toróide}
A implementação do toróide não interfere na modelagem. // TO DO

\subsection{Visão}
O framework que usamos para o jogo foi o Allegro que utiliza por trás o
OpenGL. Escolhemos essa engine por alguns motivos: \\
\begin{itemize}
    \item Ainda é utilizado nos dias de hoje pela comunidade, o que implica em mais suporte;
    \item Há bastante material didático na internet;
    \item Usa OpenGL;
\end{itemize}
A parte do código que trata a apresentação na tela foi desenvolvida no arquivo "draw.c", visando a modularização.

\subsection{Allegro}
\subsubsection{Instalação}
Para instalar o Allegro, siga as instruções da página oficial: \\ \\
    $https://wiki.allegro.cc/index.php?title=Getting_Started\#Installing\_Allegro$

\subsubsection{Memory Leak}
Detectamos que o Allegro apresenta um memory leak constante e pequeno, natural da sua implementação e que não interfere na execução do programa.

\subsubsection{Funcionamento}
// TO DO


\subsection{Controladores}
Por ora, não fizemos nenhum trabalho que envolva controladores.

\section{Makefile}
Criamos um Makefile para o projeto tal que apenas os arquivos que sofreram alguma mudança sejam recompilados. Além disso, fizemos uma especificação para deletar os arquivos .o e outra para executar o programa. \\ 
Como em nosso grupo haviam usuários de Mac OS e de Linux, fizemos um Makefile que detecta em conta o sistema operacional e compila de acordo.\\
Incluímos três flags para importar as bibliotecas do Allegro. \\ \\
Para compilar, utilize o Makefile com: 
\\
\indent $>\,make$ \\ \\
Para excluir os arquivos .o:
\\
\indent $>\,make\,clear$ \\ \\ 
Para executar uma entrada de teste pré-definida:
\\
\indent $>\,make\,test$


\section{Testes}
Criamos uma pasta "Samples" que contém diversos casos de testes distindos e verificamos, para cada um deles, se houve qualquer vazamento de memória com o Valgrind \\
Para usar o Valgrind: 
\\ \\
\indent $>\,valgrind\,--leak-check=yes\,--track-origins=yes\,./Spacewar\,50000\,<\,Teste.txt$

\section{Considerações Finais}

\end{document}\documentclass{article}
\usepackage[utf8]{inputenc}

\title{Relatório - Fase II}
\author{
    Germano Huning Neuenfeld 9298340
    \\
    Lucas Moreira Santos 9345064
    \\
    Victor Wichmann Raposo 9298020
}
\date{Março 2016}

\begin{document}

\maketitle

\section{Funcionamento}
(Nada foi alterado nessa fase) \\
O programa está projetado para receber as entradas conforme as instruções passadas pelo professor.
As forças, acelerações, posição e velocidade são tratadas como Vector para facilitar o manuseio e operações.
O projeto foi bem dividido em classes, com algumas abstrações como Body e Vector. \\ \\
Para compilar, utilize o Makefile com:
\begin{center}
$>\,make$
\end{center}
Para rodar:
\begin{center}
$>\, ./Spacewar\,(dt\,em\,inteiro)\,<\,(arquivo\,de\,entrada)$
\end{center}
\quad Exemplo:
\begin{center}
$>\,./Spacewar\,100\,<\,Teste.txt$
\end{center}

\section{Orientação a Objetos}
(Nada foi alterado nessa fase) \\
Fizemos o uso de orientação a objetos usando structs. Definimos as seguintes classes:
    \begin{itemize}
    \item $Body\,(radius;\,weight;\,position<Vector>;\,force<Vector>;\,velocity\\<Vector>)$
    \item $Vector\,(x;\,y)$
    \item $Projectile\,(duration;\,body<Body>)$
    \item $Ship\,(name;\,body<Body>)$
    \end{itemize}
Como não há orientação a objetos nativos e portanto não há herança, fizemos algo semelhante
fazendo que o os projeteis e naves tivessem uma propriedade body. O planeta é criado como um Body (por ora).

\section{Testes}
(Nada foi alterado nessa fase embora os testes tenham sido refeitos) \\
Para verificar se houve qualquer vazamento de memória, utilizamos o Valgrind.\\ \\ 
Para usar o Valgrand:
\begin{center}
$>\,valgrind\,--leak-check=yes\,--track-origins=yes\,./Spacewar\,50000\,<\,Teste.txt$
\end{center}

\section{Desenhando os Poligonos}
Para calcular os pontos dos poligonos das naves, foi criada uma função "getShipVertex" que computa os pontos dos vértices do triângulo equilátero (representação gráfica escolhida pelo grupo para a nave) dado sua posição e seu ângulo. Para o cálculo desses vértices, foi usado a matriz de rotação. Um cálculo semelhante foi efetuado nos projéteis pela função "getProjectileVertex".

\section{Considerações}
Para essa fase, não consideramos colisões. Nas naves, o raio é considerado como o lado do triângulo equilátero. Nos projéteis, o raio é utilizado na construção dos lados do retângulo que formam o desenho do projétil.
As impressões na saída padrão foram removidas.

\end{document}
