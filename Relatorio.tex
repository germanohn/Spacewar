\documentclass{article}
\usepackage[utf8]{inputenc}
\usepackage{hyperref}

\title{Relatório - Fase II}
\author{
    Germano Huning Neuenfeld 9298340
    \\
    Lucas Moreira Santos 9345064
    \\
    Victor Wichmann Raposo 9298020
}
\date{Maio 2016}

\begin{document}

\maketitle
\section{Introdução}
O documento visa descrever o projeto do jogo Spacewar. Assim, baseado na arquitetura de software MVC, queremos neste documento explicar as ideias e implementações
da primeira (modelo físico adotado e funcionamento das estruturas) e segunda parte (parte gráfica e biblioteca Allegro utilizada).

\section{Classes}
Fizemos o uso de orientação a objetos usando structs. Definimos algumas classes que serão explicadas abaixo
cuja documentação de cada função está especificada no cabeçalho das respectiva classe.

\subsection{Body}
O Body funciona como base para outros corpos. Ele possuí propriedades básicas que todo corpo (nave, planeta, projétil) possui.
Os atributos implementados nele são: raio (double), peso (double), ângulo em radianos (double), posição (Vector), força (Vector) e velocidade (Vector).

\subsection{Projectile}
Essa é a representação dos projéteis. Os atributos que eles possuem são: corpo (Body) e duração (double).
Como não há orientação a objetos em C e portanto não há hierarquia, inserimos o corpo como atributo da struct.\\
Fizemos uma lista ligada de projéteis para representar os projéteis ativos pois a conseguimos fazer a inserção e remoção em tempo constante ($O(1)$).
Logo, isso é mais eficiente do que se usássemos um vetor.

\subsection{Ship}
Essa é a representação das naves. Os atributos que eles possuem são: corpo (Body) e nome (array de chars).
Como não há orientação a objetos em C e portanto não há hierarquia, inserimos o corpo como atributo da struct.

\subsection{Vector}
Essa é a representação de um vetor no $V^2$. Os atributos que eles possuem são: x (double) e y (double).

\section{MVC}
Utilizamos a arquitetura de software Model-View-Controller para desenvolvimento do projeto por facilitar o desenvolvimento em módulos.

\subsection{Modelos}
Procuramos implementar uma orientação a objetos no nosso projeto pois acreditamos que fosse simplificar o desenvolvimento e manutenção do
código. \\
O objeto Vector que criamos simplificou muito o manuseio e cálculo de velocidades, acelerações e posições. \\
O arquivo "simulation.c" é responsável por aplicar a modelagem física, computando as forças e movimentos.

\subsubsection{Gravidade}
A cada interação zeramos os vetores de força de todos os corpos e depois aplicamos as forças gravitacionais entre:
\begin{itemize}
\item Planeta e nave 1;
\item Planeta e nave 2;
\item Nave 1 e nave 2;
\end{itemize}

\subsubsection{Movimento}
Assumimos que, para o dt dado, o movimento dos corpos é um movimento uniformemente variado. A partir disso calculamos o espaço e a velocidade dos corpos em relação ao eixo x e ao eixo y;

\subsubsection{Toróide}
Como requisitado, modelamos a tela como uma superfície toroidal. Assim, quando um objeto ultrapassa a linha superior, sua posição é alterada para a mesma posição em x
na linha inferior e vice-versa, e quando um objeto ultrapassa a linha da extremidade direita, sua posição é alterada para a mesma posição em y na linha da extremidade
esquerda e vice-versa. O vetor velocidade em qualquer um dos quatro casos não é alterado, e o vetor força respeita a modelagem a partir da nova posição do objeto.
Então a modificação efetiva diz respeito apenas à posição do objeto.

\subsubsection{Colisäo}
Para detectar as colisões entre os corpos, determinamos, para cada um, uma circunferência circunscrita no corpo. Dessa maneira, cada vez que atualizamos as posicões dos objetos verificamos se houve colisão, ou seja, se a distância entre eles é menor que a soma dos respectivos raios.

\subsection{Visão}
O framework que usamos para o jogo foi o Allegro que utiliza por trás o OpenGL. Escolhemos essa engine por alguns motivos: \\
\begin{itemize}
    \item Ainda é utilizado nos dias de hoje pela comunidade, o que implica em mais suporte;
    \item Há bastante material didático na internet;
    \item Usa OpenGL;
\end{itemize}
A parte do código que trata a apresentação na tela foi desenvolvida no arquivo "draw.c", visando a modularização.

\subsection{Allegro}
\subsubsection{Instalação}
Para instalar o Allegro, siga as instruções da página oficial: \\ \\
\url{https://wiki.allegro.cc/index.php?title=Getting\_Started#Installing\_Allegro}

\subsubsection{Memory Leak}
Detectamos que o Allegro apresenta um memory leak constante e pequeno, natural da sua implementação e que não interfere na execução do programa.

\subsubsection{Funcionamento: Parte Gráfica e Allegro}
Na segunda parte do projeto desenvolvemos a parte gráfica do jogo. Para isso, dividimos a implementação em duas funções: drawInit () e drawScene ().

\begin{itemize}
  \item drawInit: Nessa função inicializamos a imagem de cada objeto do sistema (nave1, nave2, projéteis, planeta, background). Para isso, cada objeto foi declarado como
uma variável ALLEGRO\_BITMAP e então cada objeto teve a sua imagem carregada com a função: al\_load\_bitmap ("imagemDoObjeto.png").

  \item drawScene: Nessa função desenvolvemos a simulação propriamente dita. Para isso, usamos uma ideia muito útil na Allegro chamada eventos. Eventos avisam
que algo aconteceu no programa. No caso da simulação, isso se torna muito útil, pois queremos saber quando devemos atualizar o sistema e evitar a cada momento
estar verificando se algo mudou no programa. Assim, utilizamos um "timer" (que a cada 1/dt avisa que é preciso atualizar a imagem) como um Event Source, isto é,
uma fonte de eventos que avisa ao programa que algo aconteceu, e uma Event Queue, uma fila onde os eventos (no caso timer) são inseridos, e são removidos quando usados
pelo programa. Além disso, utilizamos a variável "display", responsável pela amostragem da tela e, por padrão, vale lembrar que a Allegro cria duas imagens de buffer,
uma que se mostra para o usuário e outra que se constrói a próxima imagem a ser mostrada.
\end{itemize}
Dada então essa disposição, cada vez que o timer avisa que um novo update deve ser feito, atualiza-se as posições de cada objeto, limpa-se a tela, desenha-se o background,
desenha-se os objetos e quando todo o update está pronto, chama-se a função "al\_flip\_display", função que faz um swap das imagens do buffer, isto é, transforma a imagem
recém montada na nova imagem exibida para o usuário, enquanto o buffer da imagem antes emitida agora será o utilizado para gerar a pŕoxima imagem.

\subsection{Controladores}
Para fazer o controle das naves, usamos a leitura do teclado fornecida pela biblioteca do Allegro e integramos esse input na event queue. Para cada nave, temos um vetor de booleanos que representa as teclas de movimento. Cada vez que uma tecla for pressionada o elemento do vetor correspondente recebe valor "true" asssim quando executarmos a funçäo "UpdateKeys", a açäo a qual a tecla corresponde será executada. \\
Há uma difrença na implementaçäo da criaçäo dos progéteis. Para conseguirmos atirar um progétil a cada vez que clicamos atribuimos "true" para a variável apenas na transição da tecla do estado não pressionado para o pressionado. 

\section{Makefile}
Criamos um Makefile para o projeto tal que apenas os arquivos que sofreram alguma mudança sejam recompilados. Além disso, fizemos uma especificação para deletar os arquivos .o e outra para executar o programa. \\
Como em nosso grupo haviam usuários de Mac OS e de Linux, fizemos um Makefile que detecta em conta o sistema operacional e compila de acordo.
Inserimos também uma especificação que gera automaticamente esse relatório, caso o arquivo em .tex seja alterado (make doc) e outra que gera o zip com os arquivos relevantes para entregar no PACA (make zip).
Em razão do Allegro, tivemos que incluir três flags para importar as bibliotecas necessárias. \\ \\
Para compilar, utilize o Makefile com:
\\
\indent $>\,make$ \\ \\
Para excluir os arquivos .o:
\\
\indent $>\,make\,clear$ \\ \\
Para executar com um dt pré-definida:
\\
\indent $>\,make\,play$


\section{Testes}
Criamos uma pasta "Samples" que contém diversos casos de testes distindos e verificamos, para cada um deles, se houve qualquer vazamento de memória com o Valgrind. \\ \\
Para usar o Valgrind:
\\
\indent $>\,valgrind\,--leak-check=yes\,--track-origins=yes\,./Spacewar\,50000$

\section{Considerações Finais}
Em geral, achamos a Allegro uma implementação boa e eficiente da parte gráfica, porém ao rodar o Valgrind tivemos alguns problemas de leak de memória referentes a
funções desta biblioteca que é possível arrumar. \\
\\
Para essa fase retiramos a entrada de informações pela entrada padrão, as informações necessárias atribuimos no próprio código.
\\
Fizemos uma mudança na tecla que fazem as nave atirarem, a da nave controlada pelas teclas WAD é: "TAB"; e a da nave controlada pelas setas é: ",".
\end{document}
