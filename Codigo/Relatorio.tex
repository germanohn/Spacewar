\documentclass{article}
\usepackage[utf8]{inputenc}

\title{Relatório - Fase II}
\author{
    Germano Huning Neuenfeld 9298340
    \\
    Lucas Moreira Santos 9345064
    \\
    Victor Wichmann Raposo 9298020
}
\date{Março 2016}

\begin{document}

\maketitle

\section{Funcionamento}
(Nada foi alterado nessa fase) \\
O programa está projetado para receber as entradas conforme as instruções passadas pelo professor.
As forças, acelerações, posição e velocidade são tratadas como Vector para facilitar o manuseio e operações.
O projeto foi bem dividido em classes, com algumas abstrações como Body e Vector. \\ \\
Para compilar, utilize o Makefile com:
\begin{center}
$>\,make$
\end{center}
Para rodar:
\begin{center}
$>\, ./Spacewar\,(dt\,em\,inteiro)\,<\,(arquivo\,de\,entrada)$
\end{center}
\quad Exemplo:
\begin{center}
$>\,./Spacewar\,100\,<\,Teste.txt$
\end{center}

\section{Orientação a Objetos}
(Nada foi alterado nessa fase) \\
Fizemos o uso de orientação a objetos usando structs. Definimos as seguintes classes:
    \begin{itemize}
    \item $Body\,(radius;\,weight;\,position<Vector>;\,force<Vector>;\,velocity\\<Vector>)$
    \item $Vector\,(x;\,y)$
    \item $Projectile\,(duration;\,body<Body>)$
    \item $Ship\,(name;\,body<Body>)$
    \end{itemize}
Como não há orientação a objetos nativos e portanto não há herança, fizemos algo semelhante
fazendo que o os projeteis e naves tivessem uma propriedade body. O planeta é criado como um Body (por ora).

\section{Testes}
(Nada foi alterado nessa fase embora os testes tenham sido refeitos) \\
Para verificar se houve qualquer vazamento de memória, utilizamos o Valgrind.\\ \\ 
Para usar o Valgrand:
\begin{center}
$>\,valgrind\,--leak-check=yes\,--track-origins=yes\,./Spacewar\,50000\,<\,Teste.txt$
\end{center}

\section{Desenhando os Poligonos}
Para calcular os pontos dos poligonos das naves, foi criada uma função "getShipVertex" que computa os pontos dos vértices do triângulo equilátero (representação gráfica escolhida pelo grupo para a nave) dado sua posição e seu ângulo. Para o cálculo desses vértices, foi usado a matriz de rotação. Um cálculo semelhante foi efetuado nos projéteis pela função "getProjectileVertex".

\section{Considerações}
Para essa fase, não consideramos colisões. Nas naves, o raio é considerado como o lado do triângulo equilátero. Nos projéteis, o raio é utilizado na construção dos lados do retângulo que formam o desenho do projétil.
As impressões na saída padrão foram removidas.

\end{document}